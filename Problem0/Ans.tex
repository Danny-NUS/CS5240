\documentclass{article}

% Language setting
% Replace `english' with e.g. `spanish' to change the document language
\usepackage[english]{babel}

% Set page size and margins
% Replace `letterpaper' with`a4paper' for UK/EU standard size
\usepackage[letterpaper,top=2cm,bottom=2cm,left=3cm,right=3cm,marginparwidth=1.75cm]{geometry}

% Useful packages
\usepackage{amsmath}
\usepackage[colorlinks=true, allcolors=blue]{hyperref}

\usepackage{amsfonts}
\usepackage{amssymb,amsthm,amsmath}
\usepackage[shortlabels]{enumitem}
\usepackage{setspace}
\usepackage{tikz}
\usepackage{bbm}
\usepackage{multicol}
\usepackage{cancel}
\usepackage{graphicx}
\usepackage{booktabs}
\usepackage{extarrows}
\usepackage{mathrsfs}
\usepackage{algorithm}
\usepackage{algorithmic}


\title{Answers for Problem 0}
\author{Junchuan Zhao}

\begin{document}
\setstretch{1.2} 
\maketitle

\section{Exercise 1}
(a) A(B+C) = AB + AC\\
The $(i,j)$-th element of the left hand side is:\\
\begin{equation}
    \begin{aligned}
        [A(B+C)]_{ij} & = \sum\limits_{k}[A]_{ik}[B+C]_{kj}\\
        & = \sum\limits_{k}[A]_{ik}[B]_{kj} + [A]_{ik}[C]_{kj}\\
        & = [AB]_{ij} + [AC]_{ij}\\
        & = [AB + AC]_{ij}
    \end{aligned}
    \label{f1}
\end{equation}
Hence, A(B+C) = AB + AC\\

\noindent
(b) (AB)C = A(BC) AB $\neq$ BA\\
The $(i,j)$-th element of the left hand side is:\\
\begin{equation}
    \begin{aligned}
        [(AB)C]_{ij} & = \sum\limits_k[AB]_{ik}[C]_{kj}\\
        & = \sum\limits_k\sum\limits_m[A]_{im}[B]_{mk}[C]_{kj}\\
        & = \sum\limits_m[A]_{im}\sum\limits_k[B]_{mk}[C]_{kj}\\
        & = \sum\limits_m[A]_{im}[BC]_{mj} = [A(BC)]_{ij}
    \end{aligned}
    \label{f2}
\end{equation}
Hence, (AB)C = A(BC)\\

\noindent
(c) (A+B)$^T$ = A$^T$ + B$^T$\\
The $(i,j)$-th element of the left hand side is:\\
\begin{equation}
    \begin{aligned}
        [(A+B)^T]_{ij} & = [A+B]_{ji}\\
        & = [A]_{ji} + [B]_{ji}\\
        & = [A^T]_{ij} + [B^T]_{ij}\\
        & = [A^T + B^T]_{ij}
    \end{aligned}
\end{equation}    
Hence, (A+B)$^T$ = A$^T$ + B$^T$\\

\noindent
(d) (AB)$^T$ = B$^T$A$^T$\\
The $(i,j)$-th element of the left hand side is:\\
\begin{equation}
    \begin{aligned}
        [(AB)^T]_{ij} & = [AB]_{ji}\\
        & = \sum\limits_{k}[A]_{jk}[B]_{ki}\\
        & = \sum\limits_{k}[B^T]_{ik}[A^T]_{kj}\\
        & = [B^TA^T]_{ij}
    \end{aligned}
\end{equation}  
Hence, (AB)$^T$ = B$^T$A$^T$

\section{Exercise 2}
(a) show that the sum of the diagonal elements of the outer product xx$^T$ is equal to ${\Vert x \Vert}^2$\\
The sum of the diagonal elements of the outer product is:\\
\begin{equation}
    \begin{aligned}
        \sum\limits_{i=1}^m [xx^T]_{ii} & = \sum\limits_{i=1}^m [x]_{i1}[x]_{i1}\\
        & = \sum\limits_{i=1}^m [x]_{i}^2\\
        & = {\Vert x \Vert}^2
    \end{aligned}
\end{equation} 

\noindent
(b) AA$^T$ = nR\\
The $(i,j)$-th element of AA$^T$ is:\\
\begin{equation}
    \begin{aligned}
        [AA^T]_{ij} & = \sum\limits_{k=1}^n [A]_{ik} [A^T]_{kj}\\
        & = \sum\limits_{k=1}^n [A]_{ik} [A]_{jk}\\
        & = \sum\limits_{k=1}^n [A]_{ik} [A]_{jk}\\
    \end{aligned}
\end{equation} 
The $(i,j)$-th element of R is:\\
\begin{equation}
    \begin{aligned}
        [R]_{ij} & = \frac{1}{n}\sum\limits_{k=1}^n [x_k]_{i} [x_k]_{j}\\
        & = \frac{1}{n}\sum\limits_{k=1}^n [A]_{ik} [A]_{jk}\\
    \end{aligned}
\end{equation}
\noindent
Since, $$[AA^T]_{ij} = n[R]_{ij}$$
Hence, AA$^T$ = nR\\

\noindent
(c) BB$^T$ = nC\\
The $(i,j)$-th element of BB$^T$ is:\\
\begin{equation}
    \begin{aligned}
        [BB^T]_{ij} & = \sum\limits_{k=1}^n [B]_{ik} [B^T]_{kj}\\
        & = \sum\limits_{k=1}^n [B]_{ik} [B]_{jk}\\
        & = \sum\limits_{k=1}^n [B]_{ik} [B]_{jk}\\
    \end{aligned}
\end{equation}
The $(i,j)$-th element of C is:\\
\begin{equation}
    \begin{aligned}
        [C]_{ij} & = \frac{1}{n}\sum\limits_{k=1}^n [x_k-\mu]_{i} [x_k-\mu]_{j}\\
        & = \frac{1}{n}\sum\limits_{k=1}^n [B]_{ik} [B]_{jk}\\
    \end{aligned}
\end{equation}
Hence, BB$^T$ = nC\\

\noindent
(d) C = R - $\mu\mu^T$\\
The $(i,j)$-th element of C is:\\
\begin{equation}
    \begin{aligned}
        [C]_{ij} & = \frac{1}{n}\sum\limits_{k=1}^n [x_k-\mu]_{i} [x_k-\mu]_{j}\\
        & = \frac{1}{n}\sum\limits_{k=1}^n ([x_k]_i[x_k]_j + [\mu]_i[\mu]_j - [x_k]_i[\mu]_j - [x_k]_j[\mu]_i)\\
        & = \frac{1}{n}\sum\limits_{k=1}^n [x_k]_i[x_k]_j + [\mu]_i[\mu]_j - \frac{1}{n}\sum\limits_{k=1}^n ([x_k]_i[\mu]_j + [x_k]_j[\mu]_i)\\
        & = \frac{1}{n}\sum\limits_{k=1}^n [x_k]_i[x_k]_j + [\mu]_i[\mu]_j - 2 [\mu]_i[\mu]_j\\
        & = \frac{1}{n}\sum\limits_{k=1}^n [x_k]_i[x_k]_j - [\mu]_i[\mu]_j\\
    \end{aligned}
\end{equation}
The $(i,j)$-th element of R is:\\
\begin{equation}
    \begin{aligned}
        [R]_{ij} = \frac{1}{n}\sum\limits_{k=1}^n [x_k]_i[x_k]_j\\
    \end{aligned}
\end{equation}
The $(i,j)$-th element of R - $\mu\mu^T$ is:\\
\begin{equation}
    \begin{aligned}
        [R - \mu\mu^T]_{ij} = \frac{1}{n}\sum\limits_{k=1}^n [x_k]_i[x_k]_j - [\mu]_i[\mu]_j\\
    \end{aligned}
\end{equation}
Hence, C = R - $\mu\mu^T$\\

\noindent
(e) x$^T$Wx = $\sum\limits_{j=1}^m w_j{x_j}^2$\\
The $k$-th element of Wx is:\\
\begin{equation}
    \begin{aligned}
        [Wx]_{k} & = \sum\limits_{i} [W]_{ki}[x]_{i}\\
        & {\rm when\ i = k}, [W]_{ik} = w_k\\
        & = w_kx_k
    \end{aligned}
\end{equation}
\begin{equation}
    \begin{aligned}
        x^TWx & = \sum\limits_{k} [x^T]_{1k}[Wx]_{k1}\\
        & = \sum\limits_{k} [x]_{k}[Wx]_{k}\\
        & = \sum\limits_{k} [x]_{k}\cdot[w]_{k}[x]_{k}\\
        & = \sum\limits_{k=1}^m [w]_{k}{[x]_{k}}^2\\
    \end{aligned}
\end{equation}
Hence, $x^TWx = {\Vert W^{1/2}x \Vert}^2 = \sum\limits_{j=1}^mw_j{x_j}^2$ 
\end{document}
